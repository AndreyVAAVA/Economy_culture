\documentclass[a4paper, 12pt]{article}
\usepackage{cmap}
\usepackage[T2A]{fontenc}
\usepackage[utf8]{inputenc}
\usepackage[english, russian]{babel}
\usepackage{amsmath, amsfonts, amsthm, mathtools}
\usepackage{icomma}
\usepackage{euscript}
\usepackage{mathrsfs}
\usepackage{hyperref}
\usepackage{xcolor}
\usepackage{setspace}
\usepackage{csquotes}
\usepackage{graphicx}
\usepackage{comment}
\usepackage{array}
\graphicspath{{images/}}
\author{Волков Андрей Александрович}
\title{Экономическая культура(Лекция)}
\date{\today}

\begin{document}
	\maketitle
	Будович Лидия Сергеевна
	
	\href{budovich@mail.ru}{budovich@mail.ru}
	
	\part*{Лекция №1}
	
	\section{Экономика и экономические явления в жизини человека}
	
	\quad \, Благами называется всё то, что может удовлетворять потребности людей, приносить им пользу. Экономическая деятельность связана с процессами производства, распредления и обмена, а также употребления. В этой деятельности участвую три основных субъекта:
	\begin{itemize}
		\item Домашнее хозяйства
		\item Бизнес
		\item Государство
	\end{itemize}

	Ресурс - благо для производственной деятельности компании.
	
	Благосостояние граждан - основная цель государства.
	
	Рыночная цена - определяется соотношением предложения и спроса.
	
	Неявные издержки - альтернативные издержки
	
	Работа и отдых - учётов альтернативных издержек
	
	Пропорции деления доходов на потребление и на сбережения определяются в зависимости от:
	\begin{itemize}
		\item Величины текущего дохода
		\item Перспектив изменения дохода в будущем
		\item Предпочтения сегодняшнего потребления по сравнению с будущим
	\end{itemize}

	Ограниченно-рациональное решение - решение принимаемое взвешанно, но есть некоторые ограничения, которые влияют на выбор. (к примеру - время)


	Функции денег:
	\begin{itemize}
		\item Мера стоимости
		\item Средство обращения
		\item Средство платежа
		\item Средство накопления
		\item Мировые деньги
	\end{itemize}

	Стоимость:
	\begin{itemize}
		\item Приведённая
		\item Нынешняя
		\item Будущая
	\end{itemize}

	Ставка дисконтирования - это расчётная величина, которая позволяет оценить доходы будущих инвестиций.
	
	Фиатные деньги(так и будет, так и должно быть - перевод)

	\part*{Лекция №2}
	
	Экономика(от греч. oikonomike)
	
	Экономика - это деятельность людей, связанная с обеспечением материальных условий жизни.
	
	Экономика - наука о выборе в условиях ограниченных ресурсов.
	
	Производство - это сфера непосредственного создания ценностей
	
	Труд - целенаправленная деятельность людей по созданию жизненных благ.
	
	\part*{Лекция №3}
	
	Собственность - это система экономических и правовых отношений между людьми по поводу владения, распоряжения и использования жизненных благ.
	
	Типы собственности:
	\begin{itemize}
		\item Общая собственность
		\item Частная собственность
		\item Смешанная собственность
	\end{itemize}
	
	Система экономических отношений собственности:
	\begin{itemize}
		\item Отоношения присвоения
		\item Отношения отчуждения
		\item Отношение хозяйственного использования имущества
		\item Отношение реализации собственности(арендная плата и фиксированные платежи или доля прибыли)
	\end{itemize}

	Типы собственности:
	\begin{itemize}
		\item Владение
		\item Распоряжение
		\item Пользование
	\end{itemize}

	Виды собственности:
	\begin{enumerate}
		\item Физическое лицо
		\item Юридическое лицо
		\item Государство или мунципальные образования
	\end{enumerate}

	Объекты соственности:
	\begin{enumerate}
		\item Недвижимое имущеество
		\item Движимое имущество
		\item Интеллектуальная собственность
	\end{enumerate}

	Экономические агенты (хозяйствующие субъекты) - те, кто самостоятельно принимает решения, планирует и реализует в сфере хозяйственной (экономической) деятельности практические предприятия.
	
	Факторы производства - средства производства(в виде товаров и услуг), используемые в общественном производстве экономических благ. Это часть экономических ресурсов, реально вовлечённых в производство экономических благ.
	
	\part{Лекция №4}
	
	\begin{tabular}{| c | c | c | c | c |}
		\hline
		Характерная черта & Совершенная конкуренция & \multicolumn{3}{c}{Несовершенная конкуренция} \\
		\hline
		Число фирм & очень много & Много мелких фирм & Несколько (3-5 фирм) &  Одна \\
		\hline
		Тип продукта & Стандартизированный & Диффееренцированный & Стандартизированный или дифференцированный &  Уникальный, нет заменителей \\
		\hline
		Контроль над ценой  & Отсуствует & Некоторый, но в узких рамках & Ограниченный взаимной зависимостью, значителен при сговоре &  Значительный, фирма сама устанавливает рыночную цену \\
		\hline
		Условия вхождения в отрасль  & очень лёгкие, нет препятствий & Сравнительно лёгкие & Существенные препятствия &  Нет вхождения \\
		\hline
		Пример отрарслей  & Сельское хозяйство & Розничная торговля, производство одежды, обуви, аптекиаптеки, рестораны & Производство стали, автомобилей, сельхоз-машин, инвентаря и др. &  Местные предприиятия коммунального хозяйства \\
		\hline
	\end{tabular}
	
	TU(Total Utility) - общая полезность суммы благ.
	
	MU(Marginal Utility) - предельная полезность. Полезность последней единицы блага.
	
	Формы государственного регулирования:
	\begin{itemize}
		\item Административное регулирование - лицензирование, квотирование экспорта и импорта, контроль за ценами и качеством продукции
		\item Правовое регулирование - Гражданское законодательство, хозяйственное законодательство
		\item Косвенное экономическое регулирование - Субсидии, дотации, пособия, льготное кредитование, налоговые льготы.
		\item Прямое экономическое регулирование - кредитно-денежная, налоговая, валютная, внешне-экономическая политика
	\end{itemize}
	\section*{Законы Госсена}
	Первый закон - при последовательном потреблении единиц блага их предельная полезность имеет тенденцию к сокращению (закон убывающей предельной полезности)
	
	Второй закон - максимум полезности от потребления заданного набора благ потребитель получит при условии равенства предельных полезностей всех потреблённых благ.
	
	Спрос - платежеспособная потребность, выраженная количеством данного товара, которое может быть приобретено на рынке при различны уровнях  цены.
	
	Величина спроса(D) - это конкретное значение спроса при конкретной цене.
	
	Закон спроса выражает функциональную зависимость спроса от цена. Чем выше цена, тем ниже спрос и наоборот.
	
	На изменение спроса влияют:
	\begin{enumerate}
		\item Число покупателей
		\item Изменение в денежных доходах
		\item Изменение цен на другие товары, особенно на товары-заменители
		\item Изменение структуры населения
	\end{enumerate}

	Эластичность спроса - мера чувствительности спроса к изменению каких-либо факторов: насколько изменится объём спроса при изменении какого-либо фактора.
	
	Ценовая эластичность - показывает насколько чувствителен объём спроса к изменению цены:
	\begin{itemize}
		\item Эластичный спрос - небольшие изменения в цене приводят к значительным изменениям в объёме спроса
		\item Неэластичный спрос - значительные изменения в цене приводят к незначительным изменениям в объёме спроса.
	\end{itemize}

	Объём предложения - тот объём товара, который производителю выгодно продать при различных уровнях цен.
	
	Цена предложения - та цена, по которой производитель готов поставить на рынок определённый объём продукции.
	
	Формула расчёта коэффициента эластичность $E^p_s = \frac{\text{Изменение предложения(в \%)}}{\text{Изменение цены(в \%)}}$
	
	Равновесная цена($P_{PABH}$) - цена, при которой количесттво товара(услуг), предлагаемого продавцами, совпадает с количеством товара(услуг).
	
	\part*{Лекция №5}
	
	Последствия инфляции:
	\begin{center}
		\begin{tabular}{ | m{10em} | m{10em} | }
			\hline
			Отрицательные & Положительные \\
			\hline
			Обесценивание всего фонда накопления & Стимулирование деловой активности \\
			\hline
			Снижения уровня жизни населения с фиксированными доходами & Уменьшается внутригосударственный долг \\
			\hline
			Бартер & Растут налоговые доходы государства \\
			\hline
			... & ... \\
			\hline
		\end{tabular}
	\end{center}

	Денежное обращение - это состояние оборота денежных средств, которое определяется соотношением денежной и товарной массы, которые обмениваются на рынке.
	
	$\text{Д x О} = \text{Ц x Т}$
	
	\begin{itemize}
		\item Д - количество денег в обращений;
		\item О - скорость обращения денег (сколько раз за данное время денежная единица обслуживает торговые сделки);
		\item Ц - средняя цена типичной торговой сделки;
		\item Т - количество товаров.
	\end{itemize}


	Три основных варианта динамики рыночных цен:
	\begin{enumerate}
		\item Равновесная цена $\text{Ц} = \frac{\text{Д} \cdot \text{О}}{\text{Т}}$
		\item Инфляция $\text{Ц} > \frac{\text{Д} \cdot \text{О}}{\text{Т}}$
		\item Дефляция $\text{Ц} < \frac{\text{Д} \cdot \text{О}}{\text{Т}}$
	\end{enumerate}

	\begin{itemize}
		\item Инфляция - процесс при котором стоимость денег на рынке превосходит стомиость всех товаров.
		\item Дефляция - процесс, при котором стоимость всех товаров на рынке превосходит стоимость денег товаров.
		\item Девальвация  - снижение курса национальной валюты по отношению к какой-либо иностранной валюте, золоту.
		\item Стагфляция(стагнация + инфляция) - состояние экономики, при котором одновременно происходит спад производства, рост цен и безработица.
	\end{itemize}

	\begin{center}
		\begin{tabular}{ | m{10em} | m{10em} | }
			\hline
			Эффекты инфляции & Эффекты дефляции \\
			\hline
			Обесценивание всего фонда накопления & Стимулирование деловой активности \\
			\hline
			Снижения уровня жизни населения с фиксированными доходами & Уменьшается внутригосударственный долг \\
			\hline
			Бартер & Растут налоговые доходы государства \\
			\hline
			... & ... \\
			\hline
		\end{tabular}
	\end{center}

	\part*{Лекция №6}
	
	Леонид Витальевич Канторович в 1975 г. стал лауреатом премии вместе с Тьяллингом Купмансом.
	
	Сингапур
	Ли Куан Ю(1923-2015)
	ВВП по ППС был на момент 2019 года $93000 singapur\$$
	
	Сингапур:
	\begin{itemize}
		\item Численность насления, чел - $5 794 637$
		\item ВВП на душу населения, USD - $55 182,48\$$
		\item Безработица - 3,2\%
		\item Экспорт - 
	\end{itemize}

	ВВП - это показатель уровня экономической активности и качества жизни населения в отдельных странах и регионах за определённый период. ВВП на душу населения является ВВП делёное на количество житилей. Уровень и динамика данного показателя указываютна уровень и динамику экономического роста и развития страны, однако этот покащатель отражает лишь среднее значение, поэтому он не позволяет учитывать неравенство в доходах и благосостоянии населения.
	
	Этот показатель иногда испольщуетяс для приблизтельной оценки дохода на душу населения, причём последний показатель доступен реже. В таком случае используется ВВП по ППС(паритету покупательной способности)
	
	
	\part*{Практика №1}
	
	Лига по армрестлингу. 
	
	Можно устроить аналог Vendetta, только назвать encounter.(что-то на подобии, над именем надо подумать) И сделать аналог TOP 8. (возможен аналог Zloty tur для отбора спортсменов на encounter.)
	
	Требуемые финансы для организации первого турнира - $150 000 - 200 000 \$$. Для успешного продвижения надо будет в том числе договариваться с организаторами самых известных турниров(Engin Terzi, Larry Wheels, Niel Pickup, Ryan Bowen, Pro Panja League, возможно Игорь Мазуренко) и с разными известными личностями(Devon Larratt, Levan Saginashvilli, Бабкен и пр.). 
	
	Потребитель - фанаты армрестлинга и люди любящие бары.
	
	Цена - от 10\$ за PPV до 300\$ за VIP. (возможно создание более дорого пропуска с проживанием рядом со спортсменами)
	
	Организационно-правовая форма предприятия - ООО
	
	Требуемый начальный капитал - 1 000 000 \$
	
	Ожидаемый клиентопоток - 20 000 зрителей после нескольких турниров
	
	Потребуется примерно от 6 работников. (уборщица, комментатор, тот кто будет заниматься закупкой столов, арендой места и созданием контрактов, оператор камеры и судьи(2))
	
	\part*{Практика №3}
	\section{Задание №1}
	\begin{center}
		\begin{tabular}{ | c | c | }
			\hline
			1 & 11 \\
			\hline
			2 & 1 \\
			\hline
			3 & 1 \\
			\hline
			4 & 1 \\
			\hline
			5 & 1 \\
			\hline
			6 & 21 \\
			\hline
			7 & 23 \\
			\hline
			8 & 1 \\
			\hline
			9 & 15 \\
			\hline
			10 & 1 \\
			\hline
			11 & 7 \\
			\hline
			12 & 2 \\
			\hline
			13 & 12 \\
			\hline
			14 & 22 \\
			\hline
			15 & 1 \\
			\hline
			16 & 13 \\
			\hline
			17 & 17 \\
			\hline
			18 & 1 \\
			\hline
			19 & 18 \\
			\hline
			20 & 4 \\
			\hline
			21 & 14 \\
			\hline
			22 & 1 \\
			\hline
			23 & 1 \\
			\hline
			24 & 1 \\
			\hline
			25 & 5 \\
			\hline
			26 & 1 \\
			\hline
			27 & 1 \\
			\hline
			28 & 19 \\
			\hline
			29 & 25 \\
			\hline
		\end{tabular}
	\end{center}

	\section{Задание №2}
	
	\subsection{№1}
	\begin{enumerate}
		\item $50 400$
		\item $43 200$
		\item $46 174,05305669276$
		\item $41 781,8210902593$
	\end{enumerate}
	\subsection{№2}
	\begin{enumerate}
		\item $3 414 362,927519293$
		\item $920 000 + 3 200 000 = 4 120 000$
	\end{enumerate}
	\subsection{№3}
	$104 111,49417490307$
	\section{Задание №3}
	
\end{document}