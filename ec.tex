\documentclass[a4paper, 12pt]{article}
\usepackage{cmap}
\usepackage[T2A]{fontenc}
\usepackage[utf8]{inputenc}
\usepackage[english, russian]{babel}
\usepackage{amsmath, amsfonts, amsthm, mathtools}
\usepackage{icomma}
\usepackage{euscript}
\usepackage{mathrsfs}
\usepackage{hyperref}
\usepackage{xcolor}
\usepackage{setspace}
\usepackage{csquotes}
\usepackage{graphicx}
\usepackage{comment}
\graphicspath{{images/}}
\author{Волков Андрей Александрович}
\title{Экономическая культура(Лекция)}
\date{\today}

\begin{document}
	\maketitle
	Будович Лидия Сергеевна
	
	\href{budovich@mail.ru}{budovich@mail.ru}
	
	\part{Лекция №1}
	
	\section{Экономика и экономические явления в жизини человека}
	
	\quad \, Благами называется всё то, что может удовлетворять потребности людей, приносить им пользу. Экономическая деятельность связана с процессами производства, распредления и обмена, а также употребления. В этой деятельности участвую три основных субъекта:
	\begin{itemize}
		\item Домашнее хозяйства
		\item Бизнес
		\item Государство
	\end{itemize}

	Ресурс - благо для производственной деятельности компании.
	
	Благосостояние граждан - основная цель государства.
	
	Рыночная цена - определяется соотношением предложения и спроса.
	
	Неявные издержки - альтернативные издержки
	
	Работа и отдых - учётов альтернативных издержек
	
	Пропорции деления доходов на потребление и на сбережения определяются в зависимости от:
	\begin{itemize}
		\item Величины текущего дохода
		\item Перспектив изменения дохода в будущем
		\item Предпочтения сегодняшнего потребления по сравнению с будущим
	\end{itemize}

	Ограниченно-рациональное решение - решение принимаемое взвешанно, но есть некоторые ограничения, которые влияют на выбор. (к примеру - время)


	Функции денег:
	\begin{itemize}
		\item Мера стоимости
		\item Средство обращения
		\item Средство платежа
		\item Средство накопления
		\item Мировые деньги
	\end{itemize}

	Стоимость:
	\begin{itemize}
		\item Приведённая
		\item Нынешняя
		\item Будущая
	\end{itemize}

	Ставка дисконтирования - это расчётная величина, которая позволяет оценить доходы будущих инвестиций.
	
	Фиатные деньги(так и будет, так и должно быть - перевод)

	\section{Деньги}
	
	\section{}
	
	\section{}
	
\end{document}